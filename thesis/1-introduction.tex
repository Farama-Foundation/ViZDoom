
\chapter{Introduction}
\label{ch:introduction}
\section{Motivation}


Visual signals are to man one of the main sources of information about the surrounding environment.
While computers have greatly exceeded human capabilities in terms of data processing, they still do not match his ability to image analysis.
The increase in computing power and the emergence of technologies to perform general computation on graphics cards (CUDA, OpenCL) allowed to make a significant progress in this area by using machine learning methods.
Area of research on artificial intelligence, which in particular is gaining popularity is the reinforcement learning.
This approach was successfully used not only for identifying features in static images \cite{conf/cvpr/GoodrichA12}, but also in robotics for controlling objects using raw image from the camera \cite{rieijcnn12}.
However, performing these evaluations required gathering a large amount of test data, or building adequate physical test environment, which significantly limits the scope of possible research as
successful research on artificial intelligence algorithms require the existence of well-defined test environments.
In many cases these restrictions can be bypassed through the use of computer games as such environments.


Games can be considered analogous to what Formula 1 is for the automotive industry - they are test environment for new solutions in the field of artificial intelligence.
Because of the long-term development of game engines, they can generate realistic-looking worlds while maintaining good performance and
the possibility of scripting allows to define repeatable, deterministic test scenarios suited to the needs of reinforcement learning.
Therefore, the use of computer games as an approximation of the real world for research on the learning from visual information seems to be a reasonable way to speed up the development of this field. 
The practicality of using games for reinforcement learning from raw visual information was demonstrated in \cite{mnih-atari-2013}, where said method was applied for 7 Atari 2600 games from Arcade Learning Platform.
 In six of them the acheived results were better than in all of the previous approaches, and in the case of three results were better than those achievable by real players.
All games used in this study were 2D games, so depth perception was not taken into account.
The scope of their practical application is therefore smaller than for 3D games which offer a better approximation of the real world.
Among the 3D games, image closest to human perception of the world is generated by the FPS games.


FPS games have already been successfully used in research on artificial intelligence, especially the most popular such as Unreal Tournament \cite{6314567} \cite{6922494} Counter-Strike \cite{5035619} or Quake III Arena \cite{el2007hybrid}.
%NIE PODOBA MI SIĘ TO ZDANIE.
In these studies, the input data for reinforcement learning algorithm was prepared from data retrieved from the game engine, containing information about the locations of all objects of the game world, often requiring filtering out information that would be inaccessible to normal player.
The next step in research on the reinforcement learning from visual information would be to conduct research on the reinforcement learning from 3D visual information rendered by 3D games, especially FPS games.
So far, no studies have been conducted on the reinforcement learning from 3D visual information in general and there are no reports about the use of the image form 3d games for this purpose.


Currently there is lack of appropriate environment that allows using the FPS game to research on artificial intelligence algorithms, in which agents could learn from raw visual information.
This is a serious factor that is slowing down the progress of visual information-based research on reinforcement learning.
Today, starting that kind of research would require a large amount of work associated with the creation of a mechanism combining module of reinforcement learning and game engine.
The existence of such a tool would allow to conduct experiments straight on and focus on the goal of research without having to worry about the availability of the test environments.

\section{Aims and scope}


The main aim of this thesis is to create easy to use and flexible environment for research on intelligent agents that work and learn using the raw visual information generated by the engine of 3D FPS game and conduct experiments that confirm the usability of the created environment.

The goal will be achieved by meeting the following objectives:
\begin{itemize}
 \item to compare and select needed technologies
 \item to implement environment
 \item to define and implement test scenarios in the created environment
 \item to select learning algorithms for deep neural networks
 \item to conduct experiments for simple test scenarios
\end{itemize}


The created environment should be based on opensource, lightweight 3D FPS game that allows total control over game's processing and customizing resolution and rendering parameters.
There should be implemented spectator mode, in which agent observes human playing.
Support for creation of custom scenarios should be essential part of the environment. 
Designed API should be reinforcement learning friendly and implemented in C++ with bindings to Python and possibly other languages (Java, Lua etc.)
The environment should be multiplatform, focused on Linux, with Windows and OS X support.
	
\section{Thesis organization}


This thesis is structured as follows. 
Chapter~\ref{chapter:architecture} gives an overview of technologies and tools used to develop the environment and describes environment's architecture. It addresses design decisions and problems and contains result of performance tests. 
Chapter~\ref{chapter:api} presents the designed application programming interface, python wrapper and shows API's usage examples. 
Chapter~\ref{chapter:scenarios} gives definition of scenario and presents tools and methods for creating scenarios. It contains the description of designed scenarios. 
Chapter~\ref{chapter:experiments} shows methods used for conducting experiments and their results. 
Chapter~\ref{chapter:conclusions} concludes this thesis and proposes directions for future work.

\section{Contributions}
	\subsection{Engineering Project}
	\begin{description}
		\item[Michał Kempka] \hfill
			\begin{itemize}
				\item part of interface design,
				\item testing and experiments,
				\item part of python binding,
				\item python examples,
				\item scenarios creation,
				\item support for configuration files.
			\end{itemize}
		\item[Grzegorz Runc] \hfill
			\begin{itemize}
				\item generation of depth buffer
				\item support for offscreen rendering
			\end{itemize}
		\item[Jakub Toczek] \hfill
			\begin{itemize}
				\item A
				\item B
				\item C
			\end{itemize}
		\item[Marek Wydmuch] \hfill
			\begin{itemize}
				\item A
				\item B
				\item C
			\end{itemize}
	\end{description}
	
   	
	\subsection{Thesis}
	\begin{description}
		\item[Michał Kempka] \hfill
			\begin{itemize}
				\item foundations for Chapter \ref{ch:api},
				\item Chapter \ref{ch:scenarios},
				\item Chapter \ref{ch:experiment},
				\item Chapter \ref{ch:conclusions}.
			\end{itemize}
		\item[Grzegorz Runc] \hfill
			\begin{itemize}
				\item Chapter~\ref{ch:introduction}
				\item Section~\ref{sec:technologies}
			\end{itemize}
		\item[Jakub Toczek] \hfill
			\begin{itemize}
				\item A
				\item B
				\item C
			\end{itemize}
		\item[Marek Wydmuch] \hfill
			\begin{itemize}
				\item A
				\item B
				\item C
			\end{itemize}
	\end{description}
