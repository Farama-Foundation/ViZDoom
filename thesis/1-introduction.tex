
\chapter{Introduction}

\section{Motivation}
Deep Learning and Convolutional Neural Networks have become very popular in the last couple of years. DeepMind is a huge inspiration. Only 2D games have been researched so far, that's why we want to create a framework using 3D environment. Games are great for simulating 3 dimensional world and are perfect setting for Reinforcement Learning.\\
Stuff to mention:
\begin{itemize}
	\item Deep Neural Networks
	\item Visual Learning, Convolutional Nets, AI
	\item Reinforcement Learning
	\item DeppMind atari
	\item 2D and 3D games
\end{itemize}

Mr Jaśkowski's note:\\
''Ten rozdzial musi byc dobry. Luźne rozwazania do wykorzystania:\\
\\
There were: research in reinforcement learning in games (even 3d games), research in (reinforcement) learning from visual information. Little research (?) on reinforcement learning from visual information. No research on RL from 3d visual information (CHECK). No research on RL from 3d visual information in games.\\
\\
Why games? 1) Formula 1 for AI.  2) Mature engines with good performance and scripting abilities (scenarios) 3) Popular, well-known worlds, 4) Realistic worlds (graphics)
Why 3D games? Learning from visual information. DeepLearning advanced AI’s *perception* abilities.\\
\\
No software for such research => Need.''

\section{Aims and scope}

Mr Jaśkowski's note:\\
''The main aim of this thesis is to…\\
Requirements:\\
. . .''
\begin{itemize}
	\item opensource lightweight, 3d, fps game/engine,
	\item total control over game's processing,
	\item customizable resolution, rendering parameters, no-display mode etc.
	\item spectator mode (human is playing, agent is watching),
	\item custom scenarios support abd creation,
	\item reinforcement learning firendly API (state, action, reward),
	\item support for Linux, Windows, OS X, main focus on Linux,
	\item C++ core, API in python, perhaps in lua, java etc.
\end{itemize}
	
\section{Thesis organization}
Thesis structure

\section{Contributions}
	\subsection{Engineering Project}
	\begin{description}
		\item[Michał Kempka] \hfill
			\begin{itemize}
				\item part of interface design,
				\item testing and experiments,
				\item part of python binding,
				\item python examples,
				\item scenarios creation,
				\item support for configuration files.
			\end{itemize}
		\item[Grzegorz Runc] \hfill
			\begin{itemize}
				\item A
				\item B
				\item C
			\end{itemize}
		\item[Jakub Toczek] \hfill
			\begin{itemize}
				\item A
				\item B
				\item C
			\end{itemize}
		\item[Marek Wydmuch] \hfill
			\begin{itemize}
				\item A
				\item B
				\item C
			\end{itemize}
	\end{description}
	
   	
	\subsection{Thesis Writing}
	\begin{description}
		\item[Michał Kempka] \hfill
			\begin{itemize}
				\item foundations for Chapter \ref{ch:api},
				\item Chapter \ref{ch:scenarios},
				\item Chapter \ref{ch:experiment}.
			\end{itemize}
		\item[Grzegorz Runc] \hfill
			\begin{itemize}
				\item A
				\item B
				\item C
			\end{itemize}
		\item[Jakub Toczek] \hfill
			\begin{itemize}
				\item A
				\item B
				\item C
			\end{itemize}
		\item[Marek Wydmuch] \hfill
			\begin{itemize}
				\item A
				\item B
				\item C
			\end{itemize}
	\end{description}