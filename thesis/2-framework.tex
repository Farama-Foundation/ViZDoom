
\chapter{Framework Architecture}
\label{ch:architecture}
\section{Used Technologies}
\label{sec:technologies}

\subsection{3D FPS Game Engine}


VIZIA Environment was build around FPS game called Doom using it's game engine's modern version -- ZDoom. 
It was selected from among 6 other recognizable FPS games: Quake III Arena, Doom 3, Half-Life 2, Unreal Tournament 2004, Unreal Tournament and Cube.
Comparision between mentioned games was shown in Table~\ref{tab:engines}.


Lack of scripting support and unaccessible screen buffer were considered as critical factors for rejecting such game.
Also accessing engine only by SDK was considered a major drawback.
Some of the criteria could not be addressed in form of statement. becouse they were based on engine's analysis in the context of internal architecture and easiness of modification.


ZDoom based Doom met most of given requirements and 
It's ability to render graphics on processor was good basis for possibiliy of implementing offscreen rendering. 
\begin{table}[]
\centering
\caption{Overview of 3D FPS games considered as base of VIZIA environment}
\label{tab:engines}
\begin{tabular}{|r||p{1.3cm}|p{1.3cm}|p{1.3cm}|p{1.3cm}|p{1.3cm}|p{1.3cm}|p{1.3cm}|}
\hline
Game                      & Quake III: Arena & Doom  & Doom 3    & Half-Life 2 & Unreal Tournament 2004 & Unreal Tournament & Cube        \\ \hline
Game Engine               & ioquake3         & zdoom & Doom3.gpl & Source      & Unreal Engine 2        & Unreal Engine 4   & Cube Engine \\ \hline
Relase year               & 1999             & 1993  & 2003      & 2004        & 2004                   & 2016              & 2001        \\ \hline
Open Source               & \OK              & \OK   & \OK       &             &                        & \OK               & \OK         \\ \hline
Licence                   & GPLv2            & GPL   & GPLv3     & Closed      & Closed                 & Custom            & ZLIB        \\ \hline
Language                  & C                & C++   & C++       & C++         & C++                    & C++               & C++         \\ \hline
DirectX                   &                  &       &           & \OK         &                        & \OK               &             \\ \hline
OpenGL                    & \OK              & \OK   & \OK       & \OK         & \OK                    & \OK               & \OK         \\ \hline
Software Render           &                  & \OK   &           &             &                        &                   &             \\ \hline
Windows                   & \OK              & \OK   & \OK       & \OK         & \OK                    & \OK               & \OK         \\ \hline
Linux                     & \OK              & \OK   &           & \OK         & \OK                    & \OK               & \OK         \\ \hline
Mac OS                    & \OK              & \OK   & \OK       & \OK         & \OK                    & \OK               &             \\ \hline
Scripting                 &                  & \OK   &           & \OK         & \OK                    & \OK               & \OK         \\ \hline
Custom assets             & \OK              & \OK   & \OK       & \OK         & \OK                    & \OK               & \OK         \\ \hline
Map editor                & \OK              & \OK   & \OK       & \OK         & \OK                    & \OK               & \OK         \\ \hline
Multiplayer               & \OK              & \OK   &           &             & \OK                    & \OK               & \OK         \\ \hline
Engine access             & Code             & Code  & Code      & SDK         & SDK                    & SDK               & Code        \\ \hline
Small resolutions         & \OK              & \OK   & \OK       & \OK         & \OK                    & \OK               & \OK         \\ \hline
Screen access             & \OK              & \OK   & \OK       &             &                        & \OK               & \OK         \\ \hline
RAM                       & \OK              & \OK   & \OK       & \OK         & \OK                    &                   & \OK         \\ \hline
Disk space                & 70MB             & 40MB  & 2GB       & 4,5GB       & 6GB                    & \textgreater10GB  & 35MB        \\ \hline
Code complexity           & 6                & 5     & 8         & NA          & NA                     & 11                & 3           \\ \hline
Brand recognition         & 41,1             & 99    & 99        & 36,6        & 1,2                    & 1,2               & 0,1         \\ \hline
Active community          & \OK              & \OK   & \OK       & \OK         &                        & \OK               &             \\ \hline
Free original assets      &                  &       &           &             &                        & \OK               & \OK         \\ \hline
\end{tabular}
\end{table}


\subsection{Operating System}


Linux has been choosen as a target operating system for this environment.
Compilation tool-chain based on CMake and makefile allowed to automate building process and simplify project configuration.
It's stability makes it perfect platform for a long-term research and the biggest share in supercomputers market [http://www.top500.org/statistics/overtime/] proves it's computing abilities.
It is not insignificant that...

\subsection{Game Controller and API}


The basic module for control over Doom - game controller - was written in C++ with usage of Boost[LINK?] library.
It allowed to use same communication mechanisms in both ZDoom and game controller as ZDoom was written in C++.

Above that controller the higher layer of abstraction was developed -- Application Programming Interface for intelligent agents modules.
It was designed also in C++ having in mind binding in to Python.


Python API was considered as main way of controlling VIZIA Environment by users.
This language is one of the most popular in data science [http://www.kdnuggets.com/polls/2014/languages-analytics-data-mining-data-science.html] and googling phrase: "python machine learning" returns over 4.6 Millions results (as at 28.01.2016).
There are many machine learning libraries for Python such as scikit-learn[http://scikit-learn.org/stable/] or PyBrain[http://pybrain.org/].


\subsection{Map Editor and Scripting}


Creation of test environments has been made possible by using tools developed by Doom's community for maps and scripts editing: SLADE 3 and Doom Bulider 2. Those tools utilizes ACC -- compilter for Action Code Script (ACS), language, which support is build in ZDoom engine.
...
This topic was discussed further in Chapter~\ref{ch:scenarios}.

\section{Architecture}
Nice diagram (in DOOM style) with the arcitecture.
\begin{itemize}
\item Zdoom separate process.
\item Boost interprocess: shared memory to comunicate with zdoom.
\item Flow control and PLAYER vs SPECTATOR mode.
\item Warnings and exceptions.
\end{itemize}

\section{Problems and Solutions}
\begin{itemize}
\item Why shared memory and separate doom process and what it entails.
\item Why make/set action are like they are. Why action is a vector not just number.
\item Why state is copied in Python but not in cpp.
\item Zbuffer struggles.
\item Why Windows and Mac are not supported so well.
\item Why scenario is effectively divided into config file nad doom iwad file.
\item Why multiplayer is barely usable.
\end{itemize}

\section{Performance}
Table with some fps ratings and a graph.
Conclusions: it's fast enough, any reasonably good AI will be much slower during learning process.



